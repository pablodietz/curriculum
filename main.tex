\documentclass[10pt,a4paper]{altacv}
\PassOptionsToPackage{dvipsnames}{xcolor}

% Change the page layout if you need to
\geometry{left=1cm,right=9cm,marginparwidth=6.8cm,marginparsep=1.2cm,top=1.25cm,bottom=1.25cm,footskip=2\baselineskip}

% Change the font if you want to.
% If using pdflatex:
\usepackage[T1]{fontenc}
\usepackage[utf8]{inputenc}
\usepackage[default]{lato}

% Change the colours if you want to
\definecolor{LightBlue}{HTML}{0a66c2}
\definecolor{SlateGrey}{HTML}{2E2E2E}
\definecolor{DarkBlue}{HTML}{182d48}
\definecolor{LightGrey}{HTML}{666666}
\colorlet{heading}{DarkBlue}
\colorlet{accent}{DarkBlue}
\colorlet{emphasis}{SlateGrey}
\colorlet{body}{LightGrey}
\colorlet{linkcolor}{LightBlue}

% Change the bullets for itemize and rating marker76
\renewcommand{\itemmarker}{{\small\textbullet}}
\renewcommand{\ratingmarker}{\faCircle}

\usepackage[colorlinks, urlcolor=LightBlue]{hyperref}
\usepackage{ragged2e}

\begin{document}

\name{Pablo Ezequiel Dietz}
\tagline{Electronic Engineer}
\photo{3.3cm}{photo.jpg}
\personalinfo{
  \begin{tabular}{l|l}
  \phone{+54 9 11 6238-7606} & \email{\href{mailto:pablo.e.dietz@gmail.com}{pablo.e.dietz@gmail.com}} \\
  \location{Buenos Aires City, Argentina} & \linkedin{\href{https://www.linkedin.com/in/pablodietz}{linkedin.com/in/pablodietz}} \\
   & \printinfo{\faSkype}{\href{https://join.skype.com/invite/EF9clOE0df6L}{live:.cid.b5bf85d993e907f6}}
  \end{tabular}
  %\github{github.com/yourid}
  %\mailaddress{Charlone 2119 PB, C1427ABA, Buenos Aires City, Argentina}
  %\Skype{l}
  %\homepage{www.homepage.com}
  %\twitter{@twitterhandle}
  %\linkedin{linkedin.com/in/yourid}
  %% You MUST add the academicons option to \documentclass, then compile with LuaLaTeX or XeLaTeX, if you want to use \orcid or other academicons commands.
%   \orcid{orcid.org/0000-0000-0000-0000}
}

%% Make the header extend all the way to the right, if you want. 

\begin{fullwidth}
\makecvheader
\end{fullwidth}

%% Depending on your tastes, you may want to make fonts of itemize environments slightly smaller
% \AtBeginEnvironment{itemize}{\small}


%% Provide the file name containing the sidebar contents as an optional parameter to \cvsection.
%% You can always just use \marginpar{...} if you do
%% not need to align the top of the contents to any
%% \cvsection title in the "main" bar.
\cvsection[sidebar]{PROFILE}
\justifying

Electronic Engineer graduated at \href{http://www.fi.uba.ar/}{Buenos Aires University}, Argentina. During my studies I have acquired deep knowledge about semiconductor devices and developed special interest for telecommunications and radio-frequency applications. \\

%I joined a \href{https://csc.conicet.gov.ar/research/microwave-laboratory/}{research group} in October 2016 that it was working on UWB systems. I was a member of the team designing a general purpose UWB transceiver. Within that group, I had the opportunity to work with RF circuit design experts, and be exposed to new design techniques. \\

%Finally, in 2019, I finished my M. Sc. Thesis named “Design, implementation and characterization of an analog front-end for a general purpose UWB receiver” at that research group.\\

%Since August 2018 I have been working as modeling engineer at Allegro Microsystems, in one of the company’s R&D team. My work consists of characterization and simulation of micro-electronic devices (MOSFET, BJT, Capacitors, GMR/TMR, Hall sensors, among others) using commercial simulation software and python scripting. Additionally, I had to learn how to make wafer electrical and magnetic measurements with various techniques and automate the instruments with my own scripts. One of my most important achievements was a setup design to measure flicker noise in MOSFET transistors, produced by charge carrier traps located at silicon oxide interface.

Since August 2018 I have been working as modeling engineer at \href{https://www.allegromicro.com/}{Allegro Microsystems}, in one of the company’s R\&D team. My work consists of characterization and simulation of micro-electronic devices (MOSFET, BJT, Capacitors, GMR/TMR, Hall sensors, among others) using commercial simulation software and python scripting. Additionally, I had to learn how to make wafer electrical and magnetic measurements with various techniques and automate the instruments with my own scripts. One of my most important achievements was a setup design to measure flicker noise in MOSFET transistors, produced by charge carrier traps located at silicon oxide interface.

\medskip
\cvsection{Experience}
\nocite{*}
\justifying
\cvevent{\textbf{\href{https://www.allegromicro.com/}{Allegro Microsystems Argentina S.A.}}}{Modeling Engineer}{Aug 2018 -- present}{Buenos Aires City, Argentina}
\begin{itemize}
    \item Characterization and modeling of semiconductor devices
    \item Python scripting for visualization and data analysis
    \item Generation of automatic reports with Python
    \item Analog circuits simulation using commercial simulation software
    \item Lab. experience: DC, AC, magnetic and noise measurements on wafer
\end{itemize}

\small{
\textbf{Technologies:} CMOS, HVCMOS, Bipolar, TMR/GMR, Hall Sensors \\
\textbf{Key achievement:} Flicker Noise setup development and CMOS parasitic bipolar transistors modeling.
}\\\\
\cvtag{numpy}\cvtag{pandas}\cvtag{matplotlib}\cvtag{xarray}\cvtag{dask}\cvtag{pyside2}\cvtag{Jupyter}\cvtag{Git}\cvtag{GitLab}\cvtag{Matlab} \\

\divider

\cvevent{\textbf{\href{http://www.biodiagnostico.com.ar/}{Biodiagnóstico S.A.}}}{Field Service Engineer}{Aug 2015 -- Jul 2018}{Buenos Aires City, Argentina}
\begin{itemize}
    \item Repair and maintenance of clinical analysis instruments
    \item Client support of 5+ lines of instruments
    \item Certified courses in Argentina, Colombia, Chile, Italy and Uruguay
\end{itemize}
\small{
\textbf{Technologies:} ELISA, ANA, IFA, HPLC, Real Time PCR \\
\textbf{Brands:} BioRad, Inova, ELITechGroup, Siemens, Beckman Coulter \\
\textbf{Key achievement:} Virtualization of old operative systems that don’t run in modern PCs. \\
}

\end{document}
